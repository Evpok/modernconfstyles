%!TEX encoding = UTF-8 Unicode
%!TEX TS-program = xelatex
%
% File CLARIN2018.tex
%
% Adapted for modern XeLaTeX and Unicode by desmedt@uib.no
% Otherwise contact: jwagner@computing.dcu.ie
%%
%% Based on the style files for CLARIN2015, which were, in turn,
%% Based on the style files for CAC2014, which were, in turn,
%% Based on the style files for ACL-2014, which were, in turn,
%% Based on the style files for ACL-2013, which were, in turn,
%% Based on the style files for ACL-2012, which were, in turn,
%% based on the style files for ACL-2011, which were, in turn,
%% based on the style files for ACL-2010, which were, in turn,
%% based on the style files for ACL-IJCNLP-2009, which were, in turn,
%% based on the style files for EACL-2009 and IJCNLP-2008...

%% Based on the style files for EACL 2006 by
%%e.agirre@ehu.es or Sergi.Balari@uab.es
%% and that of ACL 08 by Joakim Nivre and Noah Smith

\documentclass{clarin}
\usepackage{hyperref}

\usepackage[backend=biber, style=apa, bibencoding=utf8]{biblatex}
\addbibresource{clarin-sample.bib}

\usepackage{covington} % if needed, for linguistic examples

\title{Instructions for CLARIN2018 Proceedings}

% - - - - - - - IMPORTANT - - - - - - -
% Leave the author information empty until your paper has been accepted

\author{First Author \\
    Department (optional)\\
    University of City, Country \\
    {\tt email@domain} \\
    \And % if needed: this makes a second column
    Second Author \\
    Department (optional)\\
    University Name without city \\
    City, Country \\
    {\tt email@domain} \\
    % \AND % if needed: this makes a second row
    % Third Author \\
    % Department (optional)\\
    % University of City, Country \\
    % {\tt email@domain} \\\And
    % Fourth Author \\
    % Department (optional)\\
    % University Name without city \\
    % City, Country \\
    % {\tt email@domain} \\
}

\begin{document}
\maketitle
\begin{abstract}
  This document contains the instructions for preparing a paper for CLARIN2018.
  The document itself
  conforms to its own specifications, and is therefore an example of
  what your manuscript should look like. These instructions should be
  used both for submissions to be reviewed \emph{and} for final versions of
  accepted papers to be published in the proceedings.
  Authors are asked to conform to all the directions
  given in this document and to other instructions in the \emph{Call for Papers}.
\end{abstract}

\section{Introduction} \label{intro}

% - - - - - - - IMPORTANT - - - - - - -
% The following footnote without marker is needed for the camera-ready
% version of the paper.
% Comment out the instructions (first text) and uncomment the 8 lines
% under "final paper" for your variant of English.
%
\blfootnote{
    %
    % for review submission
    %
    %\hspace{-0.65cm}  % space normally used by the marker
This work is licenced under a Creative Commons Attribution 4.0 International Licence.
Licence details:
\url{http://creativecommons.org/licenses/by/4.0/}
    % % final paper: en-uk version (to license, a licence)
    %
    % \hspace{-0.65cm}  % space normally used by the marker
    % This work is licensed under a Creative Commons
    % Attribution 4.0 International Licence.
    % Licence details:
    % \url{http://creativecommons.org/licenses/by/4.0/}
    %
    % % final paper: en-us version (to licence, a license)
    %
    % \hspace{-0.65cm}  % space normally used by the marker
    % This work is licenced under a Creative Commons
    % Attribution 4.0 International License.
    % License details:
    % \url{http://creativecommons.org/licenses/by/4.0/}
}

The following instructions are directed to all authors preparing a paper
for CLARIN2018.
Some of these instructions differ slightly depending on whether one is submitting
a version for initial reviewing, one is uploading an updated paper ahead of the conference, or one is uploading a final full paper for inclusion in the proceedings after the conference.
Such differences will be clearly indicated in the remainder of this document.
All authors are required to adhere to these specifications,
as well as to the instructions in the \emph{Call for Papers}.

%Authors from countries in which access to word-processing systems is limited should contact clarin@clarin.eu as soon as possible.

\section{General Instructions}

All papers should be written in English, using either American or British spelling, and should be sumitted in Portable Document Format (PDF).

Since reviewing will be double-blind, author names should not appear in any submission to be reviewed (but should be given in the final accepted paper).
Likewise, submissions to be reviewed must not mention contributors, project names, grant numbers, and names or URLs of resources or tools that have only been made publicly available in
the last 3 weeks or are about to be made public.
More detailed instructions on anonymisation will be given below.

The length of the manuscript, when written in the specified layout,
should conform to the minimum and maximum page limits described in the \emph{Call for Papers}.
There may be different page limits for extended abstracts before the conference and for full papers to be published in the Proceedings after the conference.
Papers that do not conform to the specified length and formatting requirements may be rejected without review.
Do not number the pages.


\subsection{Electronically Available Resources}

We strongly prefer that you prepare your PDF files using LaTeX with
the official CLARIN2018 style file (CLARIN2018.sty) and bibliography style
(acl.bst). These files are available at the website, where you will also find the document
you are currently reading (CLARIN2018.pdf) and its LaTeX source code
(CLARIN2018.tex).\footnote{\url{http://clarin.eu/news/clarin-annual-conference-call-papers}}

In practice, you can make a copy of this source code file (CLARIN2018.tex) and keep the preamble and basic structure but replace all content.

You can alternatively use Open Office or Microsoft Word to produce your PDF file.
In this case, we strongly recommend the use of the Open Office template file
(CLARIN2018.ott) or the Microsoft Word template
(CLARIN2018.dotx) respectively, which are also available on the website.
Reviewing will be double blind.
Therefore, if you use these templates, we suggest that you anonymise
  your source file so that the pdf produced does not retain your
  identity.  This can be done by removing any personal information
from your source document properties.


\subsection{Providing Electronic Manuscripts in PDF} \label{sect:pdf}

All versions of the papers must be uploaded in Adobe's
Portable Document Format (PDF). PDF files are usually produced from
LaTeX using the \textit{pdflatex} command. If your version of
LaTeX produces Postscript files, you can convert these into PDF
using \textit{ps2pdf} or \textit{dvipdf}. On Windows, you can also use
Adobe Distiller to generate PDF.

Please make sure that your PDF file includes all the fonts
with the necessary glyphs
(especially symbols and non-Latin
characters). When you print or create the PDF file, there is usually
an option in your printer setup to include none, all or just
non-standard fonts.  Please make sure that you select the option of
including ALL the fonts. Before sending it, test your PDF by
  printing it from a computer different from the one where it was
  created. Moreover, some word processors may generate very large PDF
files, where each page is rendered as an image. Such images may
reproduce poorly. In this case, try alternative ways to obtain the
PDF. One way on some systems is to install a driver for a postscript
printer, send your document to the printer specifying ``Output to a
file'', then convert the file to PDF.

\section{Layout} \label{ssec:layout}

It is of utmost importance to specify the A4 paper size (210~mm
x 297~mm). This is normally achieved by
specifying the \texttt{a4paper} option  in the document class.
When working with {\tt dvips}, one could also specify {\tt -t a4}.

Typeset manuscripts with a single column to a page.
The margins should be 25~mm on all sides, so that
the dimensions of the text area on an A4 page are 160~mm x 247~mm.
Manuscripts should be typed single-spaced.
Please make sure the text is hyphenated correctly.


\subsection{Fonts}

For reasons of uniformity, the Times, Times Roman or Times New Roman font should be
used for normal text. This can be accomplished by putting the necessary commands
in the preamble of a LaTeX document. Two ways are possible, depending on whether you
use traditional LaTeX2e or the more modern XeLaTeX which supports
full Unicode directly in the source. In the latter case you make sure that your source is written
in UTF-8 and you specify at least the following in the preamble:

\begin{quote}
\begin{verbatim}
\usepackage{xltxtra}
\setmainfont[Mapping=tex-text]{Times}
\end{verbatim}
\end{quote}

In traditional LaTeX, you specify at least the following instead:

\begin{quote}
\begin{verbatim}
\usepackage{times}
\usepackage{latexsym}
\end{verbatim}
\end{quote}
 If Times Roman is unavailable, use Computer
 Modern Roman (LaTeX2e's default).  Note that the latter is about
 10\% less dense than Times.

Table \ref{font-table} specifies the sizes and styles to be used for
different kinds of text elements.

\begin{table}[htb]
\begin{center}
\begin{tabular}{|l|rl|}
\hline \bf Type of Text & \bf Font Size & \bf Style \\ \hline
paper title & 15 pt & bold \\
author names & 12 pt & bold \\
author affiliation & 12 pt & \\
the word ``Abstract'' & 12 pt & bold \\
section titles & 12 pt & bold \\
subsection titles & 11 pt & bold\\
normal text & 11 pt  &\\
emphasis & & italic\\
captions & 11 pt & \\
sub-captions & 9 pt & \\
%abstract text & 10 pt & \\
bibliography & 10 pt & \\
footnotes & 9 pt & \\
\hline
\end{tabular}
\end{center}
\caption{\label{font-table} Font guide. }
\end{table}

\subsection{The First Page}
\label{ssec:first}

The title, author names and addresses should be \emph{identical}
to those entered to the electronical paper submission website in order
to maintain the consistency of author information among all
publications of the conference. If they are different, the publication
chairs may resolve the difference without consulting with you; so it
is in your own interest to double-check that the information is
consistent.

Centre the title, author's name(s) and affiliation(s) across
the page.
Do not use footnotes for affiliations. Do not include the
paper ID number which may be assigned during the submission process.

{\bf Title}: Place the title centred at the top of the first page, in
a 15 pt bold font. (For a complete guide to font sizes and styles,
see Table~\ref{font-table}) Long titles should be typed on two lines
without a blank line intervening.
Do not format title and other headings in all
capitals, except for proper names (such as ``BLEU'') that are
conventionally in all capitals.

{\bf Authors and addresses}: Centred on the page,
below the title and a blank line, place the author's names(s),
and the affiliation and addresses on subsequent lines.
Use full first (given) names (middle initials are allowed). Do
not format surnames in all capitals (e.g., use ``Schlangen'' not
``SCHLANGEN'').

The affiliation should contain the
author's complete address, and an electronic mail
address. Start the body of the first page 75~mm from the top of the
page.

In the anonymous version submitted for initial review, your real names and addresses
should \emph{not} be filled out here (see also the General Instructions section above).

{\bf Abstract}: Type the abstract between addresses and main body.
The width of the abstract text should be
smaller than main body by about 6~mm on each side.
Centre the word {\bf Abstract} in a 12 pt bold
font above the body of the abstract. The abstract should be a concise
summary of the general thesis and conclusions of the paper. It should
be no longer than 200 words. The abstract text should be in normal font size.

{\bf Text}: Begin typing the main body of the text immediately after
the abstract, observing the single-column format as shown in
the present document. Indent when starting a new paragraph. Use 11 pt for normal text.

{\bf Licence}: Include a licence statement as an unmarked (unnumbered)
footnote on the first page of the final, camera-ready paper.
See Section~\ref{licence} below for details and motivation.


\subsection{Sections}

{\bf Headings}: Use numbered sections (with Arabic
numerals) in order to facilitate cross references. Number subsections
with the section number and the subsection number separated by a dot,
in Arabic numerals. Avoid subsubsections.  Use 12 pt for section headings and 11 pt for subsection headings.

{\bf Acknowledgements}: Do not include an acknowledgements section
when submitting your paper for review, only in the final accepted version of the paper.
The acknowledgements section should go immediately before the references and its heading should not be numbered.

\textbf{References}: Gather the full set of references together under
the heading {\bf References}; place the section before any Appendices,
unless they contain references. Arrange the references alphabetically
by first author, rather than by order of occurrence in the text.
Provide as complete a citation as possible, using a consistent format,
such as the one for {\em Computational Linguistics\/} or the one in the
{\em Publication Manual of the American
% Psychological Association\/}~\cite{APA:83}.  Use of full names for
% authors rather than initials is preferred.  A list of abbreviations
% for common computer science journals can be found in the ACM
% {\em Computing Reviews\/}~\cite{ACM:83}.

The LaTeX and BibTeX style files provided roughly fit the
American Psychological Association format, allowing regular citations,
short citations and multiple citations (see also below).

{\bf Appendices}: Appendices, if any, directly follow the text and the
references (but see above).  Letter them in sequence and provide an
informative title: {\bf Appendix A. Title of Appendix}.

\section{Other Text Elements}

\subsection{Citations} \label{sec:citations}

Citations within the text appear in parentheses
as~\cite{Gusfield:97} or, if the author's name appears in the text
itself, as Gusfield~\shortcite{Gusfield:97}.  Append lowercase letters
to the year in cases of ambiguity.  Treat double authors as
in~\cite{Aho:72}, but write as in~\cite{Chandra:81} when more than two
authors are involved. Collapse multiple citations as
in~\cite{Gusfield:97,Aho:72}. Also refrain from using full citations in parentheses
as sentence constituents. For instance, instead of the following:
\begin{quote}
  ``\cite{Gusfield:97} showed that ...''
\end{quote}
we suggest you use the following form:
\begin{quote}
``\newcite{Gusfield:97}  showed that ...''
\end{quote}

If you are using the provided LaTeX and BibTeX style files, you
can use the command \verb|\newcite| to get ``author (year)'' citations.

As reviewing will be double-blind, the submitted version of the papers
should not include the authors' names and affiliations. Furthermore,
self-references that reveal the author's identity, e.g.,
\begin{quote}
``We previously showed \cite{Gusfield:97} ...''
\end{quote}
should be avoided. Instead of using the first person, use third person citations, e.g. one of the following:
\begin{quote}
``Gusfield \shortcite{Gusfield:97}
previously showed ... ''

``It has previously been shown ... \cite{Gusfield:97}''
\end{quote}
Please do not try to anonymize the citations themselves.

\subsection{Footnotes}

Put footnotes at the bottom of the page in 9 pt
text. They are preferably numbered but can also be referred to by asterisks or other
symbols.\footnote{This is how a footnote should appear.} Footnotes
should be separated from the text by a line.\footnote{Note the line
separating the footnotes from the text.}

\subsection{Graphics}

{\bf Figures and tables}: Place figures (screenshots, photographs, diagrams etc.)
and tables in the paper near where they are first discussed,
rather than at the end, if possible.

If you use colour in illustrations, make sure that sufficient contrast is maintained
when printed in black ink. Try this by printing your document on a printer
with only black ink.

Narrow graphics in the single-column format may lead to
large empty spaces,
see for example the wide margins on both sides of Table~\ref{font-table}.
If you have multiple graphics with related content, it may be
preferable to combine them in one graphic.
You can identify the sub-graphics with sub-captions below the
sub-graphics numbered (a), (b), (c) etc.
The LaTeX packages wrapfig, subfig, subtable and/or subcaption
may be useful.


{\bf Captions}: Provide a caption for every illustration and table; number each one
sequentially in the form:  ``Figure 1. Caption of the Figure.'' or ``Table 1.
Caption of the Table.'' as illustrated in Table \ref{font-table}.
Type the captions of the figures and
tables below the body, using normal 11 pt text.


\subsection{Licence Statement}
\label{licence}

To avoid the need for sending signed copyright hand-over forms to the
organisers or publishers, we require that authors licence the final version
of accepted papers under a
Creative Commons Attribution 4.0 International Licence
(CC-BY).
This means that authors (or their employers) retain copyright but
grant everybody
the right to adapt and re-distribute their paper
as long as the authors are credited and modifications listed. %\footnote{See \url{http://creativecommons.org/licenses/by/4.0/} for the licence terms.}

Depending on whether you use American or British English in your
paper, please include one of the following as an unmarked
(unnumbered) footnote on page 1 of the \emph{final} version of your paper.
The LaTeX style file (CLARIN2018.sty) adds a command
\texttt{blfootnote} for this purpose, and usage of the command is
prepared in the LaTeX source code (CLARIN2018.tex) at the start
of Section~\ref{intro} ``Introduction''.

\begin{itemize}
    %
    % % final paper: en-uk version (to license, a licence)
    %
    \item  This work is licensed under a Creative Commons
           Attribution 4.0 International Licence.
           Licence details:
           \url{http://creativecommons.org/licenses/by/4.0/}
    %
    % % final paper: en-us version (to licence, a license)
    %
    \item  This work is licenced under a Creative Commons
           Attribution 4.0 International License.
           License details:
           \url{http://creativecommons.org/licenses/by/4.0/}
\end{itemize}


We cannot accept the restrictions ``NonCommercial'' (NC) and
``ShareAlike'' (SA) as they would
limit our options for publishing and printing,
nor the restriction ``NoDerivatives'' (ND) as it may impede our
ability to add page numbers and a footer, as well as our ability
to combine papers into volumes.
Page numbers and proceedings footer are added by the organizers.
%Mentioning in the footnote that we, the organisers, add page numbers and proceedings footer is a requirement stemming from the CC-BY licence terms.

%
% We are not doing the following for CAC2014. Maybe next time.
%
% \section{XML conversion and supported \LaTeX\ packages}
%
% CAC2014 innovates over earlier years in that we will attempt to
% automatically convert your \LaTeX\ source files to machine-readable
% XML with semantic markup. This will facilitate future research that
% uses the ACL proceedings themselves as a corpus.
%
% We encourage you to submit a ZIP file of your \LaTeX\ sources along
% with the camera-ready version of your paper. We will then convert them
% to XML automatically, using the LaTeXML tool
% (\url{http://dlmf.nist.gov/LaTeXML}). LaTeXML has \emph{bindings} for
% a number of \LaTeX\ packages, including the ACL 2014 stylefile. These
% bindings allow LaTeXML to render the commands from these packages
% correctly in XML. For best results, we encourage you to use the
% packages that are officially supported by LaTeXML, listed at
% \url{http://dlmf.nist.gov/LaTeXML/manual/included.bindings}



\subsection{Translation of non-English Terms and Examples}

It is advised to supplement non-English characters and terms
with appropriate transliterations and/or translations.
Inline transliteration or translation can be represented in
the following order: original transliteration ``translation''.

Linguistic examples in languages other than English should normally be glossed
and provided with an translation in English on separate lines, as illustrated in (\ref{dutch}).
Several \LaTeX\ packages support glossed examples.

\begin{examples}
\item \label{dutch}
\gll Dit is een Nederlands voorbeeld-je.
This is a Dutch example-DIM.
\glt `This is a small example in Dutch.'
\glend
\end{examples}


\section*{Acknowledgements}

\emph{Do not include an acknowledgements section when submitting your paper for review!}

This document has been adapted from the instructions for the
CAC2014 proceedings, which are, in turn based on those for the
COLING 2014 proceedings, which are, in turn, based on the ACL-2014
proceedings compiled by Alexander Koller and Yusuke Miyao,
which are, in turn, based on the instructions for earlier ACL proceedings,
including those for ACL-2012 by Maggie Li and Michael
White, those from ACL-2010 by Jing-Shing Chang and Philipp Koehn,
those for ACL-2008 by Johanna D. Moore, Simone Teufel, James Allan,
and Sadaoki Furui, those for ACL-2005 by Hwee Tou Ng and Kemal
Oflazer, those for ACL-2002 by Eugene Charniak and Dekang Lin, and
earlier ACL and EACL formats. Those versions were written by several
people, including John Chen, Henry S. Thompson and Donald
Walker. Additional elements were taken from the formatting
instructions of the {\em International Joint Conference on Artificial
  Intelligence}.


\printbibliography

\end{document}
